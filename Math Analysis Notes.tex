% LaTeX document for mathematical analysis notes with images
\documentclass[a4paper,12pt]{article}

% Essential packages for mathematical typesetting and images
\usepackage[utf8]{inputenc}
\usepackage[russian]{babel}
\usepackage{amsmath}
\usepackage{amssymb}
\usepackage{geometry}
\geometry{margin=1in}
\usepackage{enumitem}
\usepackage{graphicx}
\usepackage{hyperref}

% Configuring hyperref for clickable links
\hypersetup{
    colorlinks=true,
    linkcolor=blue,
    urlcolor=blue
}

% Using standard Computer Modern fonts for PDFLaTeX compatibility
\usepackage{times}

% Beginning the document
\begin{document}

% Creating the title
\title{Конспект по математическому анализу}
\author{}
\date{}
\maketitle

% Table of contents
\tableofcontents
\newpage

\section{Первообразная и неопределенный интеграл. Свойства неопределенного интеграла}

\subsection{Понятие первообразной}
% Original image: https://github.com/Eggizac/MathAnaliz/blob/main/NewImage/1,%202%20%D0%B2%D0%BE%D0%BF%D1%80%D0%BE%D1%81%20(1).png?raw=true
\includegraphics[width=\textwidth]{1_2_vopros_1.png}

\subsection{Понятие неопределенного интеграла}
% Original image: https://github.com/Eggizac/MathAnaliz/blob/main/NewImage/1,%202%20%D0%B2%D0%BE%D0%BF%D1%80%D0%BE%D1%81%20(2).png?raw=true
\includegraphics[width=\textwidth]{1_2_vopros_2.png}

\subsection{Свойства неопределенного интеграла}
% Original images:
% https://github.com/Eggizac/MathAnaliz/blob/main/NewImage/1,%202%20%D0%B2%D0%BE%D0%BF%D1%80%D0%BE%D1%81%20(4).png?raw=true
% https://github.com/Eggizac/MathAnaliz/blob/main/NewImage/1,%202%20question%20(5).png?raw=true
% https://github.com/Eggizac/MathAnaliz/blob/main/NewImage/1,%202%20quetion%20(6).png?raw=true
\includegraphics[width=\textwidth]{1_2_vopros_4.png}
\includegraphics[width=\textwidth]{1_2_question_5.png}
\includegraphics[width=\textwidth]{1_2_question_6.png}

\section{Замена переменной в неопределенном интеграле}
% Original images:
% https://github.com/Eggizac/MathAnaliz/blob/main/NewImage/question%203%20(1_).png?raw=true
% https://github.com/Eggizac/MathAnaliz/blob/main/NewImage/question%203%20(2).png?raw=true
\includegraphics[width=\textwidth]{question_3_1.png}
\includegraphics[width=\textwidth]{question_3_2.png}

\section{Интегрирование по частям неопределенного интеграла}
% Original images:
% https://github.com/Eggizac/MathAnaliz/blob/main/NewImage/question%204%20(1).png?raw=true
% https://github.com/Eggizac/MathAnaliz/blob/main/NewImage/question%204%20(2).png?raw=true
\includegraphics[width=\textwidth]{question_4_1.png}
\includegraphics[width=\textwidth]{question_4_2.png}

\section{Метод замены переменной в неопределенном интеграле}

\subsection{Теория из лекции}
% Original images:
% https://github.com/Eggizac/MathAnaliz/blob/main/images/2%20%D0%B2%D0%BE%D0%BF%D1%80%D0%BE%D1%81%20(1).png?raw=true
% https://github.com/Eggizac/MathAnaliz/blob/main/images/2%20%D0%B2%D0%BE%D0%BF%D1%80%D0%BE%D1%81%20(2).png?raw=true
\includegraphics[width=\textwidth]{2_vopros_1.png}
\includegraphics[width=\textwidth]{2_vopros_2.png}

\subsection{Примеры из Демидовича на замену переменной}
% Original image: https://github.com/Eggizac/MathAnaliz/blob/main/images/2%20%D0%B2%D0%BE%D0%BF%D1%80%D0%BE%D1%81%20(%D0%B2%D0%BD%D0%B5%D1%81%D0%B5%D0%BD%D0%B8%D0%B5%20%D0%BF%D0%BE%D0%B4%20%D0%B7%D0%BD%D0%B0%D0%BA%20%D0%B4%D0%B8%D1%84%D1%84%20%D0%BF%D1%80%D0%B8%D0%BC%D0%B5%D1%80).jpg?raw=true
\includegraphics[width=\textwidth]{2_vopros_vnesenie.jpg}

\subsection{Метод внесения множителя под знак дифференциала}
Пусть про него не спрашивают в вопросе, но не зря же он находится в одном параграфе с методом замены переменной, поэтому можно его тоже рассмотреть.

\textbf{Если говорить простым языком, то суть этого метода заключается в том, чтобы преобразовать подынтегральное выражение таким образом, чтобы какой-то его кусок являлся производной такой функции, при внесении которой под знак интеграла подынтегральное выражение станет табличным интегралом.}

\subsection{Примеры внесения множителя под знак дифференциала}
% Original image: https://github.com/Eggizac/MathAnaliz/blob/main/images/2%20%D0%B2%D0%BE%D0%BF%D1%80%D0%BE%D1%81%20(%D0%BF%D1%80%D0%B8%D0%BC%D0%B5%D1%80%D1%8B%20%D0%B7%D0%B0%D0%BC%D0%B5%D0%BD%D1%8B%20%D0%BF%D0%B5%D1%80%D0%B5%D0%BC%D0%B5%D0%BD%D0%BD%D0%BE%D0%B9).jpg?raw=true
\includegraphics[width=\textwidth]{2_vopros_primery_zameny.jpg}

\section{Интегрирование рациональных выражений. Разложение рациональной дроби на простейшие дроби. Метод неопределенных коэффициентов}

\textbf{По факту эти два вопроса являются одним вопросом}

\subsection{Рациональные дроби}
Рациональная дробь вида:
\[
R(x) = \frac{a_0x^n + a_1x^{n-1} + \dots + a_{n-1}x + a_n}{b_0x^m + b_1x^{m-1} + \dots + b_{m-1}x + b_m} = \frac{P_n(x)}{Q_m(x)}
\]
является правильной, если \( n < m \), например \(\frac{x}{x^2 + 5}\),\\
является неправильной, если \( n \geq m \), например \(\frac{x^3}{3x^2 + 7x - 4}\).

\textbf{Простая дробь} является рациональной дробью, если она является одним из следующих 4 типов:
\begin{enumerate}
    \item \[
    \frac{A}{x + a} \quad \text{или} \quad \frac{A}{x - a}, \quad (A, a \in \mathbb{R})
    \]
    \item \[
    \frac{A}{(x + a)^n} \quad \text{или} \quad \frac{A}{(x - a)^n}, \quad (A, a \in \mathbb{R}, \, n \geq 2)
    \]
    \item \[
    \frac{Mx + N}{x^2 + px + q}, \quad (M, N, p, q \in \mathbb{R}), \quad (x^2 + px + q = 0)
    \]
    \item \[
    \frac{Mx + N}{(x^2 + px + q)^n}, \quad (M, N, p, q \in \mathbb{R}), \quad (x^2 + px + q = 0), \quad (n \geq 2)
    \]
\end{enumerate}

Из каждой дроби можно взять интеграл:
\begin{enumerate}
    \item \[
    A \ln(|x + a|) + C \quad \text{или} \quad A \ln(|x - a|) + C
    \]
    \item \[
    \frac{A}{(1-n)(x + a)^{n-1}} + C \quad \text{или} \quad \frac{A}{(1-n)(x - a)^{n-1}} + C
    \]
    \item \[
    \frac{M}{2} \ln|x^2 + px + q| + \left(N - \frac{pM}{2}\right) \cdot \frac{1}{\sqrt{q - \frac{p^2}{4}}} \cdot \arctan\left(\frac{x + \frac{p}{2}}{\sqrt{q - \frac{p^2}{4}}}\right) + C
    \]
    \item \[
    I_n = \frac{t}{a^2(n-1)(t^2 + a^2)^{n-1}} - \frac{1}{a^n} \cdot \frac{2n-3}{2n-2} \cdot I_{n-1}
    \]
\end{enumerate}

\textbf{Теорема:} Правильную дробь \(\frac{P_n(x)}{Q_m(x)}\), где \( Q_m(x) = (x - a)^n \cdot (x - b)^m \cdot \dots \cdot (x^2 + px + q)^s \), можно разложить на сумму простейших дробей:
\[
\frac{P_n(x)}{Q_m(x)} = \frac{A_1}{x - a} + \frac{A_2}{(x - a)^2} + \dots + \frac{A_k}{(x - a)^k} + \frac{B_1}{x - b} + \frac{B_2}{(x - b)^2} + \dots + \frac{B_l}{(x - b)^l} + \frac{M_1 x + N_1}{x^2 + px + q} + \frac{M_2 x + N_2}{(x^2 + px + q)^2} + \dots + \frac{M_s x + N_s}{(x^2 + px + q)^s}
\]

\subsection{Метод неопределенных коэффициентов}
Если дробь правильная (\( n < m \)), разложить знаменатель на множители.\\
Если дробь неправильная (\( n \geq m \)), выделить целую часть, чтобы \(\frac{R(x)}{Q(x)}\) была правильной.

Алгоритм:
\begin{enumerate}
    \item Разложить дробь на простейшие:
    \[
    \frac{P_n(x)}{Q_m(x)} = \frac{A}{x - a} + \frac{B}{x - b} + \frac{Cx + D}{x^2 + px + q}
    \]
    \item Привести к общему знаменателю.
    \item Сравнять числители.
    \item Решить СЛАУ для коэффициентов.
\end{enumerate}

% Original image: https://github.com/Eggizac/MathAnaliz/blob/main/images/4%20%D0%B2%D0%BE%D0%BF%D1%80%D0%BE%D1%81%20(%D0%BF%D1%80%D0%B8%D0%BC%D0%B5%D1%80%201).jpg?raw=true
\includegraphics[width=\textwidth]{4_vopros_primer_1.jpg}

\subsection{Метод частных значений}
% Original image: https://github.com/Eggizac/MathAnaliz/blob/main/images/4%20%D0%B2%D0%BE%D0%BF%D1%80%D0%BE%D1%81%20(%D0%BF%D1%80%D0%B8%D0%BC%D0%B5%D1%80%202).jpg?raw=true
\includegraphics[width=\textwidth]{4_vopros_primer_2.jpg}

\section{Интегрирование иррациональных выражений}

Интеграл вида:
\[
\int R\left(x, \sqrt{x^{m_1}}, \sqrt{x^{m_2}}, \dots\right) dx = \left| x = t^S, \quad S - \text{общий знаменатель дробей } \frac{m_1}{n_1}, \frac{m_2}{n_2} \right| = \int R\left(t^S, \frac{Sm_1}{n_1}, \frac{Sm_2}{n_2}, \dots\right) S t^{S-1} dt
\]

% Original image: https://github.com/Eggizac/MathAnaliz/blob/main/images/6%20%D0%B2%D0%BE%D0%BF%D1%80%D0%BE%D1%81%20(%D0%BF%D1%80%D0%B8%D0%BC%D0%B5%D1%80%D1%8B).jpg?raw=true
\includegraphics[width=\textwidth]{6_vopros_primery.jpg}

\section{Интегрирование дифференциального бинома. Подстановки Эйлера}

Дифференциальный бином:
\[
\int x^m (a + b x^n)^p \, dx, \quad (a, b \in \mathbb{R}, \, m, n, p \in \mathbb{Q})
\]

Подстановки Эйлера:
\begin{enumerate}
    \item Если \( a > 0 \): \[
    a + b x^n = t^k, \quad x = \left( \frac{t^k - a}{b} \right)^{1/n}, \quad dx = \frac{k t^{k-1}}{n b} \left( \frac{t^k - a}{b} \right)^{1/n - 1} dt
    \]
    \item Если \( b > 0 \): \[
    a + b x^n = t x^k
    \]
    \item \[
    a + b x^n = t^2 x^k
    \]
\end{enumerate}

\section{Интегрирование тригонометрических функций}

\subsection{Способы нахождения}
\begin{enumerate}
    \item Использование тригонометрических формул:
    % Original image: https://github.com/Eggizac/MathAnaliz/blob/main/images/8%20%D0%B2%D0%BE%D0%BF%D1%80%D0%BE%D1%81%20(1%20%D0%BF%D1%80%D0%B8%D0%BC%D0%B5%D1%80).jpg?raw=true
    \includegraphics[width=\textwidth]{8_vopros_1_primer.jpg}
    \item Понижение степени:
    \[
    \sin^2(x) = \frac{1 - \cos(2x)}{2}, \quad \cos^2(x) = \frac{1 + \cos(2x)}{2}, \quad \sin(2x) = 2 \sin(x) \cos(x)
    \]
    % Original image: https://github.com/Eggizac/MathAnaliz/blob/main/images/8%20%D0%B2%D0%BE%D0%BF%D1%80%D0%BE%D1%81%20(%D0%BF%D1%80%D0%B8%D0%BC%D0%B5%D1%80%202).jpg?raw=true
    \includegraphics[width=\textwidth]{8_vopros_primer_2.jpg}
    \item Замена переменных:
    % Original image: https://github.com/Eggizac/MathAnaliz/blob/main/images/8%20%D0%B2%D0%BE%D0%BF%D1%80%D0%BE%D1%81%20(%D0%BF%D1%80%D0%B8%D0%BC%D0%B5%D1%80%203).jpg?raw=true
    \includegraphics[width=\textwidth]{8_vopros_primer_3.jpg}
    \item Универсальная подстановка:
    \[
    t = \tan\left(\frac{x}{2}\right), \quad \sin x = \frac{2t}{t^2 + 1}, \quad \cos x = \frac{1 - t^2}{t^2 + 1}, \quad x = 2 \arctan(t)
    \]
    % Original image: https://github.com/Eggizac/MathAnaliz/blob/main/images/8%20%D0%B2%D0%BE%D0%BF%D1%80%D0%BE%D1%81%20(%D0%BF%D1%80%D0%B8%D0%BC%D0%B5%D1%80%204).jpg?raw=true
    \includegraphics[width=\textwidth]{8_vopros_primer_4.jpg}
\end{enumerate}

\section{Определенный интеграл. Понятие интегральной суммы}
% Original images:
% https://github.com/Eggizac/MathAnaliz/blob/main/NewImage/question%2010%20(1).png?raw=true
% https://github.com/Eggizac/MathAnaliz/blob/main/NewImage/question%2010%20(2).png?raw=true
% https://github.com/Eggizac/MathAnaliz/blob/main/NewImage/question%2010%20(3).png?raw=true
\includegraphics[width=\textwidth]{question_10_1.png}
\includegraphics[width=\textwidth]{question_10_2.png}
\includegraphics[width=\textwidth]{question_10_3.png}

\section{Верхние и нижние суммы Дарбу. Свойства сумм Дарбу}
% Original images:
% https://github.com/Eggizac/MathAnaliz/blob/main/images/10%20%D0%B2%D0%BE%D0%BF%D1%80%D0%BE%D1%81%20(1%20%D1%84%D0%BE%D1%82%D0%BE).png?raw=true
% https://github.com/Eggizac/MathAnaliz/blob/main/images/10%20%D0%B2%D0%BE%D0%BF%D1%80%D0%BE%D1%81%20(2%20%D1%84%D0%BE%D1%82%D0%BE).png?raw=true
\includegraphics[width=\textwidth]{10_vopros_1_foto.png}
\includegraphics[width=\textwidth]{10_vopros_2_foto.png}

\section{Необходимое и достаточное условие интегрируемости}
% Original image: https://github.com/Eggizac/MathAnaliz/blob/main/NewImage/quetion%2012%20(1).png?raw=true
\includegraphics[width=\textwidth]{question_12_1.png}

\section{Свойства интегрируемых функций}
% Original images:
% https://github.com/Eggizac/MathAnaliz/blob/main/images/12%20%D0%B2%D0%BE%D0%BF%D1%80%D0%BE%D1%81%201.png?raw=true
% https://github.com/Eggizac/MathAnaliz/blob/main/images/12%20%D0%B2%D0%BE%D0%BF%D1%80%D0%BE%D1%81%202.png?raw=true
% https://github.com/Eggizac/MathAnaliz/blob/main/images/12%20%D0%B2%D0%BE%D0%BF%D1%80%D0%BE%D1%81%203.png?raw=true
% https://github.com/Eggizac/MathAnaliz/blob/main/images/12%20%D0%B2%D0%BE%D0%BF%D1%80%D0%BE%D1%81%204.png?raw=true
% https://github.com/Eggizac/MathAnaliz/blob/main/images/12%20%D0%B2%D0%BE%D0%BF%D1%80%D0%BE%D1%81%205.png?raw=true
\includegraphics[width=\textwidth]{12_vopros_1.png}
\includegraphics[width=\textwidth]{12_vopros_2.png}
\includegraphics[width=\textwidth]{12_vopros_3.png}
\includegraphics[width=\textwidth]{12_vopros_4.png}
\includegraphics[width=\textwidth]{12_vopros_5.png}

\section{Оценки интегралов. Теорема о среднем значении}
% Original images:
% https://github.com/Eggizac/MathAnaliz/blob/main/NewImage/question%2014%20(1).png?raw=true
% https://github.com/Eggizac/MathAnaliz/blob/main/NewImage/question%2014%20(2).png?raw=true
% https://github.com/Eggizac/MathAnaliz/blob/main/NewImage/question%2014%20(3).png?raw=true
% https://github.com/Eggizac/MathAnaliz/blob/main/NewImage/question%2014%20(4).png?raw=true
% https://github.com/Eggizac/MathAnaliz/blob/main/NewImage/question%2014%20(5).png?raw=true
\includegraphics[width=\textwidth]{question_14_1.png}
\includegraphics[width=\textwidth]{question_14_2.png}
\includegraphics[width=\textwidth]{question_14_3.png}
\includegraphics[width=\textwidth]{question_14_4.png}
\includegraphics[width=\textwidth]{question_14_5.png}

\section{Определенный интеграл как функция верхнего предела. Формула Ньютона-Лейбница}
% Original images:
% https://github.com/Eggizac/MathAnaliz/blob/main/images/14%20%D0%B2%D0%BE%D0%BF%D1%80%D0%BE%D1%81%201.png?raw=true
% https://github.com/Eggizac/MathAnaliz/blob/main/images/14%20%D0%B2%D0%BE%D0%BF%D1%80%D0%BE%D1%81%202.png?raw=true
% https://github.com/Eggizac/MathAnaliz/blob/main/images/14%20%D0%B2%D0%BE%D0%BF%D1%80%D0%BE%D1%81%203.png?raw=true
% https://github.com/Eggizac/MathAnaliz/blob/main/images/14%20%D0%B2%D0%BE%D0%BF%D1%80%D0%BE%D1%81%204.png?raw=true
% https://github.com/Eggizac/MathAnaliz/blob/main/images/14%20%D0%B2%D0%BE%D0%BF%D1%80%D0%BE%D1%81%205.png?raw=true
\includegraphics[width=\textwidth]{14_vopros_1.png}
\includegraphics[width=\textwidth]{14_vopros_2.png}
\includegraphics[width=\textwidth]{14_vopros_3.png}
\includegraphics[width=\textwidth]{14_vopros_4.png}
\includegraphics[width=\textwidth]{14_vopros_5.png}

\section{Замена переменной в определенном интеграле}
% Original images:
% https://github.com/Eggizac/MathAnaliz/blob/main/NewImage/question%2016%20(1).png?raw=true
% https://github.com/Eggizac/MathAnaliz/blob/main/NewImage/question%2016%20(2).png?raw=true
\includegraphics[width=\textwidth]{question_16_1.png}
\includegraphics[width=\textwidth]{question_16_2.png}

\section{Формула интегрирования по частям для определенного интеграла}
% Original images:
% https://github.com/Eggizac/MathAnaliz/blob/main/images/16%20%D0%B2%D0%BE%D0%BF%D1%80%D0%BE%D1%81%201.png?raw=true
% https://github.com/Eggizac/MathAnaliz/blob/main/images/16%20%D0%B2%D0%BE%D0%BF%D1%80%D0%BE%D1%81%202.png?raw=true
\includegraphics[width=\textwidth]{16_vopros_1.png}
\includegraphics[width=\textwidth]{16_vopros_2.png}

\section{Геометрические приложения определенного интеграла}

\subsection{Длина кривой}
% Original images:
% https://github.com/Eggizac/MathAnaliz/blob/main/images/17%20%D0%B2%D0%BE%D0%BF%D1%80%D0%BE%D1%81%201.png?raw=true
% https://github.com/Eggizac/MathAnaliz/blob/main/images/17%20%D0%B2%D0%BE%D0%BF%D1%80%D0%BE%D1%81%202.png?raw=true
% https://github.com/Eggizac/MathAnaliz/blob/main/images/17%20%D0%B2%D0%BE%D0%BF%D1%80%D0%BE%D1%81%203.png?raw=true
% https://github.com/Eggizac/MathAnaliz/blob/main/images/17%20%D0%B2%D0%BE%D0%BF%D1%80%D0%BE%D1%81%204.png?raw=true
% https://github.com/Eggizac/MathAnaliz/blob/main/images/17%20%D0%B2%D0%BE%D0%BF%D1%80%D0%BE%D1%81%205.png?raw=true
\includegraphics[width=\textwidth]{17_vopros_1.png}
\includegraphics[width=\textwidth]{17_vopros_2.png}
\includegraphics[width=\textwidth]{17_vopros_3.png}
\includegraphics[width=\textwidth]{17_vopros_4.png}
\includegraphics[width=\textwidth]{17_vopros_5.png}

\subsection{Площадь плоской фигуры}
См. \url{http://mathprofi.ru/vychislenie_ploshadi_c_pomoshju_opredelennogo_integrala.html}

\subsection{Площадь криволинейной трапеции}
% Original images:
% https://github.com/Eggizac/MathAnaliz/blob/main/images/18%20%D0%B2%D0%BE%D0%BF%D1%80%D0%BE%D1%81%201.png?raw=true
% https://github.com/Eggizac/MathAnaliz/blob/main/images/18%20%D0%B2%D0%BE%D0%BF%D1%80%D0%BE%D1%81%202.png?raw=true
% https://github.com/Eggizac/MathAnaliz/blob/main/images/18%20%D0%B2%D0%BE%D0%BF%D1%80%D0%BE%D1%81%203.png?raw=true
\includegraphics[width=\textwidth]{18_vopros_1.png}
\includegraphics[width=\textwidth]{18_vopros_2.png}
\includegraphics[width=\textwidth]{18_vopros_3.png}

\subsection{Объем тела вращения}
% Original images:
% https://github.com/Eggizac/MathAnaliz/blob/main/images/18%20%D0%B2%D0%BE%D0%BF%D1%80%D0%BE%D1%81%204.png?raw=true
% https://github.com/Eggizac/MathAnaliz/blob/main/images/18%20%D0%B2%D0%BE%D0%BF%D1%80%D0%BE%D1%81%205.png?raw=true
% https://github.com/Eggizac/MathAnaliz/blob/main/images/18%20%D0%B2%D0%BE%D0%BF%D1%80%D0%BE%D1%81%206.png?raw=true
\includegraphics[width=\textwidth]{18_vopros_4.png}
\includegraphics[width=\textwidth]{18_vopros_5.png}
\includegraphics[width=\textwidth]{18_vopros_6.png}

\section{Несобственный интеграл первого рода}
% Original images:
% https://github.com/Eggizac/MathAnaliz/blob/main/NewImage/quetion%2020%20(1).png?raw=true
% https://github.com/Eggizac/MathAnaliz/blob/main/NewImage/question%2020%20(2).png?raw=true
% https://github.com/Eggizac/MathAnaliz/blob/main/NewImage/question%2020%20(3).png?raw=true
\includegraphics[width=\textwidth]{question_20_1.png}
\includegraphics[width=\textwidth]{question_20_2.png}
\includegraphics[width=\textwidth]{question_20_3.png}

\section{Несобственный интеграл второго рода}
% Original images:
% https://github.com/Eggizac/MathAnaliz/blob/main/images/20%20%D0%B2%D0%BE%D0%BF%D1%80%D0%BE%D1%81%201.png?raw=true
% https://github.com/Eggizac/MathAnaliz/blob/main/images/20%20%D0%B2%D0%BE%D0%BF%D1%80%D0%BE%D1%81%202.png?raw=true
% https://github.com/Eggizac/MathAnaliz/blob/main/images/20%20%D0%B2%D0%BE%D0%BF%D1%80%D0%BE%D1%81%203.png?raw=true
\includegraphics[width=\textwidth]{20_vopros_1.png}
\includegraphics[width=\textwidth]{20_vopros_2.png}
\includegraphics[width=\textwidth]{20_vopros_3.png}

\section{Признаки сравнения. Понятие абсолютной сходимости несобственного интеграла}
% Original images:
% https://github.com/Eggizac/MathAnaliz/blob/main/NewImage/question%2022%20(1).png?raw=true
% https://github.com/Eggizac/MathAnaliz/blob/main/NewImage/question%2022%20(2).png?raw=true
% https://github.com/Eggizac/MathAnaliz/blob/main/NewImage/question%2022%20(3).png?raw=true
% https://github.com/Eggizac/MathAnaliz/blob/main/NewImage/question%2022%20(4).png?raw=true
% https://github.com/Eggizac/MathAnaliz/blob/main/NewImage/question%2022%20(5).png?raw=true
% https://github.com/Eggizac/MathAnaliz/blob/main/NewImage/question%2022%20(6).png?raw=true
% https://github.com/Eggizac/MathAnaliz/blob/main/NewImage/question%2022%20(7).png?raw=true
\includegraphics[width=\textwidth]{question_22_1.png}
\includegraphics[width=\textwidth]{question_22_2.png}
\includegraphics[width=\textwidth]{question_22_3.png}
\includegraphics[width=\textwidth]{question_22_4.png}
\includegraphics[width=\textwidth]{question_22_5.png}
\includegraphics[width=\textwidth]{question_22_6.png}
\includegraphics[width=\textwidth]{question_22_7.png}

\section{Признак Абеля и признак Дирихле сходимости несобственных интегралов}
% Original images:
% https://github.com/Eggizac/MathAnaliz/blob/main/NewImage/question%2023%20(1).png?raw=true
% https://github.com/Eggizac/MathAnaliz/blob/main/NewImage/question%2023%20(2).png?raw=true
% https://github.com/Eggizac/MathAnaliz/blob/main/NewImage/question%2023%20(3).png?raw=true
% https://github.com/Eggizac/MathAnaliz/blob/main/NewImage/question%2023%20(4).png?raw=true
\includegraphics[width=\textwidth]{question_23_1.png}
\includegraphics[width=\textwidth]{question_23_2.png}
\includegraphics[width=\textwidth]{question_23_3.png}
\includegraphics[width=\textwidth]{question_23_4.png}

\section{Пространства \( \mathbb{R}^n \) и множества в них. Функции нескольких переменных}
% Original images:
% https://github.com/Eggizac/MathAnaliz/blob/main/NewImage/question%2024%20(1).png?raw=true
% https://github.com/Eggizac/MathAnaliz/blob/main/NewImage/question%2024%20(2).png?raw=true
% https://github.com/Eggizac/MathAnaliz/blob/main/NewImage/question%2024%20(3).png?raw=true
% https://github.com/Eggizac/MathAnaliz/blob/main/NewImage/question%2024%20(4).png?raw=true
% https://github.com/Eggizac/MathAnaliz/blob/main/NewImage/question%2024%20(5).png?raw=true
% https://github.com/Eggizac/MathAnaliz/blob/main/NewImage/question%2024%20(6).png?raw=true
% https://github.com/Eggizac/MathAnaliz/blob/main/NewImage/question%2024%20(7).png?raw=true
% https://github.com/Eggizac/MathAnaliz/blob/main/NewImage/question%2024%20(8).png?raw=true
% https://github.com/Eggizac/MathAnaliz/blob/main/NewImage/question%2024%20(9).png?raw=true
% https://github.com/Eggizac/MathAnaliz/blob/main/NewImage/question%2024%20(10).png?raw=true
% https://github.com/Eggizac/MathAnaliz/blob/main/NewImage/question%2024%20(11).png?raw=true
% https://github.com/Eggizac/MathAnaliz/blob/main/NewImage/question%2024%20(12).png?raw=true
% https://github.com/Eggizac/MathAnaliz/blob/main/NewImage/question%2024%20(13).png?raw=true
% https://github.com/Eggizac/MathAnaliz/blob/main/NewImage/question%2024%20(14).png?raw=true
% https://github.com/Eggizac/MathAnaliz/blob/main/NewImage/question%2024%20(15).png?raw=true
\includegraphics[width=\textwidth]{question_24_1.png}
\includegraphics[width=\textwidth]{question_24_2.png}
\includegraphics[width=\textwidth]{question_24_3.png}
\includegraphics[width=\textwidth]{question_24_4.png}
\includegraphics[width=\textwidth]{question_24_5.png}
\includegraphics[width=\textwidth]{question_24_6.png}
\includegraphics[width=\textwidth]{question_24_7.png}
\includegraphics[width=\textwidth]{question_24_8.png}
\includegraphics[width=\textwidth]{question_24_9.png}
\includegraphics[width=\textwidth]{question_24_10.png}
\includegraphics[width=\textwidth]{question_24_11.png}
\includegraphics[width=\textwidth]{question_24_12.png}
\includegraphics[width=\textwidth]{question_24_13.png}
\includegraphics[width=\textwidth]{question_24_14.png}
\includegraphics[width=\textwidth]{question_24_15.png}

\section{Предел и непрерывность функции нескольких переменных}
% Original images:
% https://github.com/Eggizac/MathAnaliz/blob/main/NewImage/question%2025%20(1).png?raw=true
% https://github.com/Eggizac/MathAnaliz/blob/main/NewImage/question%2025%20(2).png?raw=true
% https://github.com/Eggizac/MathAnaliz/blob/main/NewImage/question%2025%20(3).png?raw=true
% https://github.com/Eggizac/MathAnaliz/blob/main/NewImage/question%2025%20(4).png?raw=true
% https://github.com/Eggizac/MathAnaliz/blob/main/NewImage/question%2025%20(5).png?raw=true
\includegraphics[width=\textwidth]{question_25_1.png}
\includegraphics[width=\textwidth]{question_25_2.png}
\includegraphics[width=\textwidth]{question_25_3.png}
\includegraphics[width=\textwidth]{question_25_4.png}
\includegraphics[width=\textwidth]{question_25_5.png}

\section{Частные производные функций нескольких переменных}
% Original images:
% https://github.com/Eggizac/MathAnaliz/blob/main/NewImage/question%2026%20(1).png?raw=true
% https://github.com/Eggizac/MathAnaliz/blob/main/NewImage/question%2026%20(2).png?raw=true
% https://github.com/Eggizac/MathAnaliz/blob/main/NewImage/question%2026%20(3).png?raw=true
\includegraphics[width=\textwidth]{question_26_1.png}
\includegraphics[width=\textwidth]{question_26_2.png}
\includegraphics[width=\textwidth]{question_26_3.png}

\section{Дифференциал функции нескольких переменных}
% Original images:
% https://github.com/Eggizac/MathAnaliz/blob/main/NewImage/question%2027%20(1).png?raw=true
% https://github.com/Eggizac/MathAnaliz/blob/main/NewImage/question%2027%20(2).png?raw=true
% https://github.com/Eggizac/MathAnaliz/blob/main/NewImage/question%2027%20(3).png?raw=true
% https://github.com/Eggizac/MathAnaliz/blob/main/NewImage/question%2027%20(4).png?raw=true
\includegraphics[width=\textwidth]{question_27_1.png}
\includegraphics[width=\textwidth]{question_27_2.png}
\includegraphics[width=\textwidth]{question_27_3.png}
\includegraphics[width=\textwidth]{question_27_4.png}

\section{Производные сложных функций. Инвариантность формы полного дифференциала}
% Original images:
% https://github.com/Eggizac/MathAnaliz/blob/main/NewImage/question%2028%20(1).png?raw=true
% https://github.com/Eggizac/MathAnaliz/blob/main/NewImage/question%2028%20(2).png?raw=true
% https://github.com/Eggizac/MathAnaliz/blob/main/NewImage/question%2028%20(3).png?raw=true
% https://github.com/Eggizac/MathAnaliz/blob/main/NewImage/question%2028%20(4).png?raw=true
% https://github.com/Eggizac/MathAnaliz/blob/main/NewImage/question%2028%20(5).png?raw=true
% https://github.com/Eggizac/MathAnaliz/blob/main/NewImage/question%2028%20(6).png?raw=true
\includegraphics[width=\textwidth]{question_28_1.png}
\includegraphics[width=\textwidth]{question_28_2.png}
\includegraphics[width=\textwidth]{question_28_3.png}
\includegraphics[width=\textwidth]{question_28_4.png}
\includegraphics[width=\textwidth]{question_28_5.png}
\includegraphics[width=\textwidth]{question_28_6.png}

\section{Частные производные высших порядков}
% Original images:
% https://github.com/Eggizac/MathAnaliz/blob/main/NewImage/question%2029%20(1).png?raw=true
% https://github.com/Eggizac/MathAnaliz/blob/main/NewImage/question%2029%20(2).png?raw=true
% https://github.com/Eggizac/MathAnaliz/blob/main/NewImage/question%2029%20(2).png?raw=true
\includegraphics[width=\textwidth]{question_29_1.png}
\includegraphics[width=\textwidth]{question_29_2.png}

\section{Дифференциалы высших порядков}
% Original images:
% https://github.com/Eggizac/MathAnaliz/blob/main/NewImage/question%2030%20(1).png?raw=true
% https://github.com/Eggizac/MathAnaliz/blob/main/NewImage/question%2030%20(2).png?raw=true
% https://github.com/Eggizac/MathAnaliz/blob/main/NewImage/question%2030%20(3).png?raw=true
% https://github.com/Eggizac/MathAnaliz/blob/main/NewImage/question%2030%20(4).png?raw=true
\includegraphics[width=\textwidth]{question_30_1.png}
\includegraphics[width=\textwidth]{question_30_2.png}
\includegraphics[width=\textwidth]{question_30_3.png}
\includegraphics[width=\textwidth]{question_30_4.png}

\end{document}